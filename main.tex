\documentclass[ja]{2019b}
% 講演者についての情報
\PresenterInfo
% 講演数(半角数字)
{1}
% 氏名
{谷口暁星}
% 氏(ひらがな)
{たにぐち}
% 名(ひらがな)
{あきお}
% 所属機関
{名古屋大学}
% 会員種別(半角英小文字)
{a}
% 会員番号(半角数字4桁)
{5892}
% メールアドレス(半角)
{taniguchi@a.phys.nagoya-u.ac.jp}
% 講演についての情報
\PaperInfo
% 記者発表(半角英小文字)
{}
% 講演分野(半角)
{R}
% 講演形式(半角英小文字)
{a}
% キーワード(5つまで)
{galaxies: ISM}
{galaxies: nuclei}
{galaxies: starburst}
{galaxies: Seyfert}
{galaxies: individual (NGC 1068, NGC 1097)}
% 題名
{衝撃波トレーサー分子の高空間分解能観測で探る活動銀河核における特異なHCN/HCO$^{+}$輝線強度比の起源}
% 氏名及び所属(複数の場合は「, 」で区切)
{谷口 暁星, 中島 拓, 田村 陽一 (名古屋大), 高野 秀路 (日本大), 濤崎 智佳 (上越教育大), 河野 孝太郎 (東京大), 原田 ななせ (ASIAA), 泉 拓磨, 今西 昌俊 (国立天文台)}
\begin{document}

活動銀河核(active galactic nucleus; AGN)や爆発的星形成(starburst; SB)などの銀河の熱源の違いを、ミリ波サブミリ波帯の輝線観測を通して診断する手法を確立することは、埋もれた銀河の活動性を近傍から遠方宇宙に渡って理解するために必要不可欠である。
現在、ALMAをはじめとする干渉計の高空間分解能観測により、HCN/HCO$^{+}$輝線強度比がAGNの核周辺円盤(circumnuclear disk; CND)で高い値($>1$)を持つことが報告されている(e.g., Kohno et~al. 2008, Garc\'{i}a-Burillo et~al. 2014, Izumi et~al. 2016a)。
一方、この特異な輝線比の起源は様々な可能性が議論されているものの、観測的な制限は十分に得られていない。

本研究では、ジェットやアウトフローの力学的加熱による高温環境下でHCNの存在量が増加した可能性 (Harada et~al. 2010, 2013) に着目した。
そこで、衝撃波トレーサー分子SiOを近傍の複数の活動銀河で観測し、X線光度に依らずSiO/HCO$^{+}$とHCN/HCO$^{+}$との間に空間的な相関が見られるかどうかを検証した。
ALMAによる高空間分解能(15--25pc)観測で、これまでにNGC1068, 1097のCNDおよびSB領域で、SiO (6--5), H$^{13}$CN (3--2), H$^{13}$CO$^{+}$ (3--2)を検出した。
その結果、SiO/H$^{13}$CO$^{+}$とH$^{13}$CN/H$^{13}$CO$^{+}$がともにCNDで高い値($\gtrsim1.5$, $\gtrsim4$)を持つことが明らかになった。
また、輻射輸送モデル計算により、これらの輝線比を再現するためには、高温($T_{\mathrm{kin}}>100$~K), 高密度($n_{\mathrm{H2}}>10^{7}$~cm$^{-3}$), かつH$^{13}$CNとSiOの存在量がともにH$^{13}$CO$^{+}$に対して増加している必要があるという制限を得た。
これらより、観測から力学的加熱の可能性を強く示唆する結果が得られた。

\end{document}
